\section{Задача}

\subsection*{Условие:}
В обращении есть монеты достоинством в $1, 2, 5, 10, 20, 50$ копеек и $1$ рубль. Известно, что $k$ 
монетами можно набрать $m$ копеек. Докажите, что $m$ монетами можно набрать $k$ рублей.

\paragraph*{Подсказка:}
Для каждой монеты достоинством в $n$ копеек есть монета достоинством в $\dfrac{100}{n}$ копеек.

\subsection*{Доказательство:}
Пусть среди $k$ монет, дающих в сумме $m$ копеек, есть $a_1$ монет по $1$ копейке, $a_2$ "--- по 
$2$ копейки, $a_3$ "--- по $5$, $a_4$ "--- по $10$, $a_5$ "--- по $20$, $a_6$ "--- по $50$ копеек и
$a_7$ "--- по $1$ рублю. Тогда

\begin {equation*}
    \left\lbrace 
        \begin {array}  {rrrrrrrcl}
            a_1 \ + &
            2 \cdot a_2 \ + &
            5 \cdot a_3 \ + &
            10 \cdot a_4 \ + &
            20 \cdot a_5 \ + &
            50 \cdot a_6 \ + &
            100 \cdot a_7 &
            = &
            m
            \\
            a_1 \ + & a_2 \ + & a_3 \ + & a_4 \ + & a_5 \ + & a_6 \ + & a_7 & = & k
        \end {array}
    \right.
\end {equation*}

Умножим второе равенство на $100$ и запишем его в виде:
$$
    100 \cdot a_1 +
    50 \cdot 2 \cdot a_2 +
    20 \cdot 5 \cdot a_3 +
    10 \cdot 10 \cdot a_4 +
    5 \cdot 20 \cdot a_5 +
    2 \cdot 50 \cdot a_6 +
    100 \cdot a_7
    =
    100 \cdot k
$$
\noindent отсюда следует, что если взять $100 \cdot a_7$ монет по $1$ копейке, $50 \cdot a_6$ "---
по $2$, $20 \cdot a_5$ "--- по $5$, $10 \cdot a_4$ "--- по $10$, $5 \cdot a_3$ "--- по $20$,
$2 \cdot a_2$ "--- по $50$ копеек и $a_1$ монет по $1$ рублю, то в сумме они дадут
$100 \cdot k$ копеек, то есть $k$ рублей. А согласно первому равенству монет будет $m$.