\section{Задача}

\subsection*{Условие:}
Докажите следующий признак делимости на $37$: <<Для того, чтобы узнать, делится ли число на $37$, 
надо разбить его справа налево на группы по три цифры. Если сумма полученных трёхзначных чисел 
делится на $37$, то и данное число делится на $37$>>.

\paragraph*{Пояснение:}
Слово <<трёхзначные>> употреблено условно: некоторые из групп могут начинаться с нулей и быть на
самом деле двузначными или меньше; не трёхзначной будет и самая левая группа, если количество цифр
нашего числа не кратно $3$.

\subsection*{Доказательство:}
Докажем, что полученная в условии сумма $S_N$ даёт тот же остаток при делении на $37$, что и
исходное число $N$. Так как число $999$ делится на $37$, достаточно доказать, что числа $N$ и $S_N$
дают одинаковый остаток при делении на $999$. Дальнейшее доказательство аналогично доказательству
признака делимости на $9$:

\begin {equation*}
    \begin {array}  {rcl}
        N - S_N & = &
        ... + \overline{a_{3r+2} a_{3r+1} a_{3r}} \cdot 10^{3r} + ... +
        \overline{a_2 a_1 a_0} - S_N
        \\
        & = &
        ... + \overline{a_{3r+2} a_{3r+1} a_{3r}} \cdot \left(10^{3r} - 1\right) + ... +
        \overline{a_5 a_4 a_3} \cdot 999 + \overline{a_2 a_1 a_0} \cdot 0
    \end {array}
\end {equation*}

\noindent каждое слагаемое в этой сумме делится на $999$, а значит, вся сумма делится на $999$.
Следовательно, числа $N$ и $S$ дают одинаковый остаток при делении на $999$.