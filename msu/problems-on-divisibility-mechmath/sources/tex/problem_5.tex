\section{Задача}

\subsection*{Условие:}
Из натурального числа $N_1$ вычли сумму его цифр $S_{N_1}$, из полученного числа
$N_2 = N_1 - S_{N_1}$ вновь вычли сумму уже его цифр $S_{N_2}$ и т. д. После $11$ таких итераций
получился ноль. Какое число $N_1$ было в начале?

\paragraph*{Подсказка:}
Разность между числом и суммой его цифр делится на $9$.

\subsection*{Решение:}
Разность между числом и суммой его цифр делится на $9$, поэтому все числа, которые мы получали, 
делились на $9$ (кроме, может быть, исходного). Пойдём с конца: ноль мог получиться из любого 
однозначного натурального числа после вычитания из него суммы цифр. Но из них на $9$ делится только 
$9$. Поэтому на предпоследнем шаге у нас было число $9$. Но $9$ можно получить только из одного
числа, делящегося на $9$, "--- из $18$. И так далее пока не дойдём до числа $81$. Тут путь 
раздваивается: $81$ можно получить и из $90$, и из $99$. Сделаем последний шаг назад (теперь
делимость на $9$ нам уже не важна!) "--- $90$ ни из какого числа получить нельзя, а для $99$ есть
целых $10$ возможных предшественников: $100, 101, 102, ..., 109$.

\subsection*{Ответ:}
Любое натуральное число от 100 до 109.