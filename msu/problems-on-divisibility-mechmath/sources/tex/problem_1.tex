\section{Задача}

\subsection*{Условие:}
Существует ли прямоугольный треугольник с одним из катетов равным $101$?

\subsection*{Решение:}
Обозначим гипотенузу как $a$, а катеты как $b$ и $c$. Тогда по теореме Пифагора:

$$
    a^2 = b^2 + c^2
$$

\noindent пусть катет $c = 101$, тогда

$$
    a^2 = b^2 + 101^2 \quad \Rightarrow \quad a^2 - b^2 = 101^2
$$

Теперь воспользуемся формулой разности квадратов:

$$
    a^2 - b^2 = (a - b)\cdot(a + b) = 1 \cdot 101 \cdot 101
$$

Из последнего равенства следует следующая система уравнений:

\begin {equation*}
    \left[
        \begin {array} {l} 
            \begin {cases}
                a - b = 1
                \\
                a + b = 101^2
            \end {cases}
            \vspace{1em} \\
            \begin {cases}
                a - b = 101^2
                \\
                a + b = 1
            \end {cases}
            \vspace{1em} \\
            \begin {cases}
                a - b = 101
                \\
                a + b = 101
            \end {cases}
        \end {array}
    \right.
    \longrightarrow\quad
    \left[
        \begin {array} {l}
            \begin {cases}
                a = b + 1
                \\
                b = \dfrac{101^2 - 1}{2} = 5100
            \end {cases}
            \vspace{1em} \\
            \begin {cases}
                a = 1 - b
                \\
                b = \dfrac{1 - 101^2}{2} < 0
            \end {cases}
            \vspace{1em} \\
            \begin {cases}
                a = 101 - b
                \\
                b = 0
            \end {cases}
        \end {array}
    \right.
\end {equation*}

Случаи $2$ и $3$ не подходят, потому что стороны треугольника должны быть положительными. Из системы
$1$ получается окончательное решение:

\begin {equation*}
    \begin {cases}
        a = 5101
        \\
        b = 5100
    \end {cases}
\end {equation*}

\subsection*{Ответ:}
Прямоугольный треугольник с одним из катетов равным $101$ существует. Второй его катет равен
$5100$, а гипотенуза "--- $5101$. 