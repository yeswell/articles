\section{Задача}

\subsection*{Условие:}
Есть $2017$ ящиков с шариками, в первом ящике $1$ шарик, во втором $2$ шарика, в $2017$-ом ящике
"--- $2017$ шариков. Иногда из какого-нибудь ящика берут два шарика и перекладывают их по одному в
два других. Может ли в какой-то момент оказаться, что каждый шарик побывал в каждом ящике ровно
один раз, а в конце вернулся в свой начальный ящик?

\subsection*{Решение:}
Во всех ящиках находится
$$
    N = \dfrac{(1 + 2017)\cdot2017}{2} = 1009\cdot2017
$$
\noindent шариков. Каждый шарик нужно переложить $K = 2017$ раз, чтобы он побывал в каждом ящике
ровно один раз, и в итоге мог оказаться в своём первоначальном ящике.\\

Отсюда легко понять, что для ситуации, описанной в условии, надо совершить
$M = N \cdot K = 1009 \cdot 2017^2$ вытаскиваний.\\

По условию всегда вытаскивают по $2$ шарика, поэтому можно наблюдать противоречие: с одной стороны,
количество вытаскиваний должно быть чётным числом, с другой стороны, мы вычислили, что число
вытаскиваний будет нечётным.

\subsection*{Ответ:}
Не может.