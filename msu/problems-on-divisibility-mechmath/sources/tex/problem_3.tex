\section{Задача}

\subsection*{Условие:}
В клетках квадратной таблицы $10 \times 10$ расставлены числа от $1$ до $100$. Пусть
$S_1, S_2, ..., S_{10}$ "--- суммы чисел, стоящих в столбцах таблицы. Могло ли оказаться так, что 
среди чисел $S_1, S_2, ..., S_{10}$ каждые два соседних различаются на $1$?

\subsection*{Решение:}
Если $S_i$ и $S_{i+1}$ различаются на $1$, то эти два числа имеют разную чётность, то есть в 
последовательности $S_1, S_2, ..., S_{10}$ чётные и нечётные числа строго чередуются. Значит, среди
чисел $S_1, S_2, ..., S_{10}$ ровно пять чётных и пять нечётных. Отсюда следует, что сумма
$S_1 + S_2 + ... + S_{10}$ нечётна. С другой стороны,
$S_1 + S_2 + ... + S_{10} = 1 + 2 + ... + 100$, а в этой сумме $50$ нечётных слагаемых, поэтому она
чётна. Возникает противоречие.

\subsection*{Ответ:}
Не могло.