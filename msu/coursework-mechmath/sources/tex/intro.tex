\section {Введение}

В данной работе рассмотрены квадратные уравнения с одним параметром, множество значений которого 
необходимо определить в зависимости от условий задачи. В качестве таковых выступают варианты
расположения корней квадратного уравнения на некотором промежутке или луче. Работа состоит из двух
частей: теоретической, с описанием и обоснованием методов решения задач и примерами решений, и
части с задачами для самостоятельного решения.\\

Всего в работе выделено шесть различных вариантов расположения корней на промежутке или луче. При
этом в качестве промежутка может выступать отрезок, полуинтервал или интервал, а луч может быть
открытым или закрытым. Подробнее о каждом варианте сказано в третьей части работы.\\

Для каждого из вариантов расположения корней представлен общий способ решения, не зависящий от типа
границ промежутка или луча, но дающий не полное решение. Также для каждого типа границ выделены и
рассмотрены все особые случаи, которые в совокупности с решением, полученным общим методом, дают
полное решение задачи.