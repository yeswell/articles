\section {Обозначения}

Для однозначного понимания текста работы ниже приведён список используемых обозначений:

\begin {enumerate} [labelindent=\parindent, leftmargin=*]
    \item {$f(x) = ax^2 + bx + c \ $ "--- \ квадратный трёхчлен,}
    \item {$p \in \mathbb{R} \ $ "--- \ параметр,}
    \item {$a = a(p), \ b = b(p), \ c = c(p) \ $ "--- \ коэффициенты квадратного трёхчлены, 
           зависящие от параметра $ p $,}
    \item {$D = b^2 - 4ac \ $ "--- \ дискриминант квадратного трёхчлена,}
    \item {$x_0 = - \dfrac{b}{2a} \ $ "--- \ координата вершины параболы $ y = f(x) $,}
    \item {$x_{1,\,2} = \dfrac{- b \pm \sqrt{D}}{2a} \ $ "--- \ корни квадратного уравнения
           $f(x) = 0 $,}
    \item {$\alpha \in \mathbb{R} \ $ "--- \ левая граница промежутка или луча,}
    \item {$\beta \in \mathbb{R} \ $ "--- \ правая граница промежутка или луча.}
\end {enumerate}