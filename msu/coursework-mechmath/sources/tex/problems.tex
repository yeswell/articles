\section {Задачи}

\subsection * {Задача 1}
Найдите все действительные значения параметра $p$, при которых уравнение
$x - 2 = \sqrt{2(p - 1) \cdot x - 1}$ имеет единственное решение.\\
\textbf{Ответ:} $p \in [\frac{3}{4}, +\infty)$.\\

\subsection * {Задача 2}
При каких значениях $p$ уравнение $2p \cdot (x + 1)^2 - |x + 1| + 1 = 0$ имеет четыре различных
корня?\\
\textbf{Ответ:} $p \in (0, \frac{1}{8})$\\

\subsection * {Задача 3}
При каких значениях параметра $p$ имеется ровно один корень у уравнения 
$(p - 1) \cdot 4^x + (2p - 3) \cdot 6^x = (3p - 4) \cdot 9^x$?\\
\textbf{Ответ:} $p \in (-\infty, 1] \cup \{\frac{5}{4}\} \cup [\frac{4}{3}, +\infty)$\\

\subsection * {Задача 4}
Найдите все значения $p$, при каждом из которых среди корней уравнения 
$p \cdot x^2 + (p + 4) \cdot x + (p + 1) = 0 $ имеется ровно один отрицательный.\\
\textbf{Ответ:} $p \in (-1, 0] \cup \{\frac{2 + 2 \sqrt{13}}{3}\}$.\\

\subsection * {Задача 5}
При каких значениях $p$ один из корней уравнения больше $x = 1$, а другой меньше?
$(p^2 + p + 1) \cdot x^2 + (2p - 3) \cdot x + (p - 5) = 0$.\\
\textbf{Ответ:} $p \in (-2-\sqrt{11}, -2+\sqrt{11})$


\subsection * {Задача 6}
Найдите все значения $p$, при которых существует единственное решение уравнения
$3\cdot\sqrt[5]{x+2} - 16p^2\cdot\sqrt[5]{32x+32} = \sqrt[10]{x^2 + 3x + 2}$.\\
\textbf{Ответ:}
$p\in(-\infty,-\frac{1}{2\sqrt2}]\cup[-\frac{1}{4},\frac{1}{4}]\cup[\frac{1}{2\sqrt2},+\infty)$\\

\pagebreak

\subsection * {Задача 7}
При каких значениях параметра $p$ уравнение имеет ровно три различных корня?
$16^x -3 \cdot 2^{3x + 1} + 2 \cdot 4^{x + 1} - (4 - 4p) \cdot 2^{x - 1} - (p^2 - 2p + 1) = 0$.\\
\textbf{Ответ:} $p \in (0, 1) \cup (1, 4) \cup (4, 5)$\\

\subsection * {Задача 8}
Найдите все $p$, при которых существует два различных корня уравнения
$(\log_2{(x+1)} - \log_2{(x-1)})^2 - 2(\log_2{(x+1)} - \log_2{(x-1)}) - (p^2 - 1) = 0$.\\
\textbf{Ответ:} $p \in (-1, 0) \cup (0, 1)$

\subsection * {Задача 9}
Найдите все значения параметра $p$, при которых оба корня квадратного уравнения
$p \cdot x^2 - (2p-1) \cdot x + (3p - 1) = 0$ больше нуля.\\

\subsection * {Задача 10}
При каких значениях параметра $p$ уравнение $\dfrac{1}{9^{|x-2|}} - \dfrac{p}{3^{|x-2|}} - 4 = 0$
имеет ровно один корень?\\

\subsection * {Задача 11}
Для уравнения $x^2 - 2x \cdot \log_p{(p + 1)} + \log_p{(p - 4)} = 0$ найдите все значения $p$, при
которых оба корня находятся внутри интервала $(0, 1)$.\\

\subsection * {Задача 12}
При каких $p$ уравнение $2\cos^2{(x)} + p^2 \cos{(x)} + 1 = 0$ не имеет решений?\\

\subsection * {Задача 13}
Необходимо найти все значения параметра $p$, при которых уравнение
$2p \cdot x^2 - 2x - (3p + 2)=0$ имеет ровно один корень на луче $(1, +\infty)$.\\

\subsection * {Задача 14}
Найдите все значения параметра $p$ при которых оба корня квадратного уравнения
$(p - 1) \cdot x^2 - (p + 1) \cdot x + p = 0$ лежат в интервале $(1, 3)$?\\

\subsection * {Задача 15}
При каких значениях параметра $p$ нет ни одного вещественного корня у уравнения 
$25^{p+x}+25^{p-x}+2(p+1)\cdot5^{2p+x}+2(p+1)\cdot5^{2p-x}= (3-p^2)\cdot5^{2p}$?\\